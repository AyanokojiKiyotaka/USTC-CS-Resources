\documentclass[11pt,a4paper]{article}
\usepackage{graphicx}
\usepackage{subfigure}
\usepackage{amsmath}
\usepackage{makecell}
\usepackage[utf8]{inputenc}
\usepackage{listings} %放代码
\usepackage{xcolor} %代码着色宏包
\usepackage{xeCJK}
\usepackage{float}

\lstset{
	backgroundcolor=\color{white}, 
	%\tiny < \scriptsize < \footnotesize < \small < \normalsize
	basicstyle = \footnotesize,       
	breakatwhitespace = false,        
	breaklines = true,                 
	captionpos = b,                    
	commentstyle = \color{black}\bfseries,
	extendedchars = false,
	frame =shadowbox, 
	framerule=0.5pt,
	keepspaces=true,
	keywordstyle=\color{black}\bfseries, % keyword style
	language = C++,                     % the language of code
	otherkeywords={string}, 
	numbers=left, 
	numbersep=5pt,
	numberstyle=\tiny\color{black},
	rulecolor=\color{black},         
	showspaces=false,  
	showstringspaces=false, 
	showtabs=false,    
	stepnumber=1,         
	stringstyle=\color{mymauve},        % string literal style
	tabsize=4,          
	title=\lstname                      
}

%好像是数学的包
\usepackage{amsmath}
\usepackage{amssymb}
\usepackage{mathrsfs}
%页面布局包
\usepackage{geometry}
%画图包
\usepackage{tikz}
%画图背景包
\usetikzlibrary{backgrounds}

\geometry{left=3.0cm, right=3.0cm, top=3cm, bottom=3cm}

%自定义命令
\newcommand{\psiG}{\psi_{G}}
%在tikz中画一个顶点
%#1:node名称
%#2:位置
%#3:标签
\newcommand{\newVertex}[3]{\node[circle, draw=black, line width=1pt, scale=0.8] (#1) at #2{#3}}
%在tikz中画一条边
\newcommand{\newEdge}[2]{\draw [black,very thick](#1)--(#2)}
%在tikz中放一个标签
%#1:名称
%#2:位置
%#3:标签内容
\newcommand{\newLabel}[3]{\node[line width=1pt] (#1) at #2{#3}}
\newcommand{\keyPoint}[2]{$\bullet$#1:#2\par}
\newcommand{\jumpLine} {\hspace*{\fill} \par}




\title{Introduction to Computing Systems\\Homework 5}
\author{PB18111697 王章瀚}

\begin{document}
	\maketitle
	\section{}
	The instruction is 0010 001 111111111. So R1 will contains \underline{x23FF}.
	
	\section{}
	The \#30 cannot be represented as a signed number in 5 bits.\par
	To fix it, we can use the ADD twice. Like this: \par
	\begin{lstlisting}[language=C++]
	ADD	R3, R3, #15
	ADD	R3, R3, #15
	\end{lstlisting}
	
	\section{}
	\subsection*{a}
	As following:\par
	\begin{tabular}{|c|c|}
		\hline 
		Symbol & Address \\ 
		\hline 
		LOOP & x3003 \\ 
		\hline 
		L1 & x300A \\ 
		\hline 
		NEXT & x300B \\ 
		\hline 
		DONE & x300D \\ 
		\hline 
		NUMBERS & x300E \\ 
		\hline 
	\end{tabular} 
	\subsection*{b}
	After the program is finished, R0 contains the amount of the numbers; R3 contains the amount of the numbers that the last bit is 0; R4 contains the amount of the numbers that the last bit is 1.\par
	
	\section{}
	\jumpLine
	(a). LDR R3, R1, \#0\par
	(b). NOT R4, R4\par
	(c). ADD R4, R4, \#1\par
	
	\section{}
	\subsection*{a}
	They will be as following:\par
	\begin{tabular}{|c|c|}
		\hline 
		R0 & x300B \\ 
		\hline 
		R1 & x300D \\ 
		\hline 
		R2 & x000A \\ 
		\hline 
		R3 & x1263 \\ 
		\hline 
		R4 & x300B \\ 
		\hline 
	\end{tabular} 
	\subsection*{b}
	They will be as following:\par
	\begin{tabular}{|c|c|}
		\hline 
		Addr1 & x300B \\ 
		\hline 
		Addr2 & x000A \\ 
		\hline 
		Addr3 & x000A \\ 
		\hline 
		Addr4 & x300B \\ 
		\hline 
		Addr5 & x300D \\ 
		\hline 
	\end{tabular} 

	\section{}
	The R2 contains the data in x3500, instead of the address x3500. So when storing the number, we shall use "\underline{LD R2, VECTOR}", then "\underline{STR R0, R2, \#1}".\par
	
	\section{}
	Count the amount of the data in which at least one 1 locates at the same position with R1(from MASK), ranging from x4000 to x4009. Then store it into x5000.
	
	\section{}
	Interrupt-driven I/O is more efficient. Because it avoids the trouble that the CPU need to keep asking whether there is an I/O in Polling Mode.\par
	
	\section{}
	\subsection*{a}
	Keep outputing x0032 in ASCII code, which is 2.\par
	\subsection*{b}
	It output the key striked in twice.\par
	\subsection*{c}
	It will still keep outputing 2.\par
	
	\section{}
	Output "ABCDEFGHIJ"(exclude the quotation marks)\par
	
	
		
\end{document}