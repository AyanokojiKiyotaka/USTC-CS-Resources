\documentclass[11pt,a4paper]{article}
\usepackage{graphicx}
\usepackage{subfigure}
\usepackage{amsmath}
\usepackage{makecell}
\usepackage[utf8]{inputenc}
\usepackage{listings} %放代码
\usepackage{xcolor} %代码着色宏包
\usepackage{xeCJK}
\usepackage{float}

%好像是数学的包
\usepackage{amsmath}
\usepackage{amssymb}
\usepackage{mathrsfs}
%页面布局包
\usepackage{geometry}
%画图包
\usepackage{tikz}
%画图背景包
\usetikzlibrary{backgrounds}

\geometry{left=3.0cm, right=3.0cm, top=3cm, bottom=3cm}

%自定义命令
\newcommand{\psiG}{\psi_{G}}
%在tikz中画一个顶点
%#1:node名称
%#2:位置
%#3:标签
\newcommand{\newVertex}[3]{\node[circle, draw=black, line width=1pt, scale=0.8] (#1) at #2{#3}}
%在tikz中画一条边
\newcommand{\newEdge}[2]{\draw [black,very thick](#1)--(#2)}
%在tikz中放一个标签
%#1:名称
%#2:位置
%#3:标签内容
\newcommand{\newLabel}[3]{\node[line width=1pt] (#1) at #2{#3}}


\title{Introduction to Computing Systems\\Homework 4}
\author{PB18111697 王章瀚}

\begin{document}
	\maketitle
	\section{}
	\textcolor{red}{
		A: BRnzp -171
		B: JSR -171
		Both A and B result in the PC being changed to (PC+1)-171.
		However, B saves the linkage information in R7 and A does not affect R7.
	}\par
	\qquad The instruction A stands for BR(Branches), while B is a JSR. After A, PC would be loaded with $(PC^{+}+SEXT(101010101))$. And after B, PC would be loaded with $(PC^{+}+SEXT(11101010101))$.\par
	The main differences is that the instruction with opcodes 0000 needs bit[11:9] to check n,z,p; while the one with opcodes 0100 needs bit[11] to check whether it is a JSR or JSRR. So the offset in BR can only get 9 bits, while in JSR, it can get to 11 bits, which means a larger range is available.\par
	
	\section{}
	\subsection*{a}
	\textcolor{red}{
		0001 011 010 1 00000 (ADD R3, R2, \#0 )
		Using only NOT.
	}\par
	\begin{lstlisting}[language=C++]
	1001 011 010 111111		; NOT R3, R2
	1001 011 011 111111		; NOT R3, R3
	\end{lstlisting}
	
	\subsection*{b}
	\begin{lstlisting}[language=C++]
	1001 001 011 111111		; NOT R1, R3
	0001 001 001 1 00001		; ADD R1, R1, #1
	0001 001 010 000 001		; ADD R1, R2, R1
	\end{lstlisting}
	\subsection*{c}
	\begin{lstlisting}[language=C++]
	0101 001 001 000 001		;AND R1, R1, R1
	\end{lstlisting}
	\subsection*{d}
	No. The condition codes in LC3 is based on the last value loaded into a general purpose register. And this value cannot be both negative and zero at the same time.\par
	\subsection*{e}
	\begin{lstlisting}[language=C++]
	0101 010 010 1 00000		;AND R2, R2, #0
	\end{lstlisting}
	
	\section{}
	\begin{lstlisting}[language=C++]
	0001 101 000 1 11000
	\end{lstlisting}
	
	\section{}
	Including PC, MEMORY(and MDR, MAR), IR, CONTROL UNIT, REG FILE, MARMUX.\par

	\section{}
	\subsection*{a}
	Turn R0 into $R0\times 2^{4}$. Or Turn R0 into $R0<<4$.
	\subsection*{b}
	\begin{table}[H]
		\footnotesize
		\begin{tabular}{|c|c|c|c|c|c|c|c|c|c|c|c|}
			\hline 
			PC & R0 & R1 & R2 & R3 & R4 & R5 & R6 & R7 & N & Z & P \\ 
			\hline 
			x3006(since there's a breakpoint) & x0050 & x0000 & x0000 & x0000 & x0000 & x0000 & x0000 & x0000 & 0 & 1 & 0 \\ 
			\hline 
		\end{tabular} 
	\end{table}
	\subsection*{c}
	\begin{tabular}{|c|c|}
		\hline 
		instruction & clock cycles \\ 
		\hline 
		AND & 1+5+1+1+1=9 \\ 
		\hline 
		ADD & 1+5+1+1+1=9 \\ 
		\hline 
		LD & 1+5+1+1+1+5+1=15 \\ 
		\hline 
		BRp & 1+5+1+1+1=9 \\ 
		\hline 
	\end{tabular} \par
	And the program uses 1 AND, (1+4*2=9) ADDs, 1 LD, 4 BRps.\par
	So the total number would be \underline{$1\times 9+9\times 9+1\times 15+4\times 9=141$}

	\section{}
	The numbers of ones in x3100 would be stored into R0.
	
	\section{}
	\begin{lstlisting}[language=C]
	0110 001 000 000000		;LD R1, R0, #0
	0000 100 000000010		;BRn
	\end{lstlisting}
	
	
	
	
		
\end{document}