\documentclass[UTF8]{article}
\usepackage{graphicx}
\usepackage{subfigure}
\usepackage{amsmath}
\usepackage{makecell}
\usepackage[utf8]{inputenc}
\usepackage[space]{ctex} %中文包
\usepackage{listings} %放代码
\usepackage{xcolor} %代码着色宏包
\usepackage{CJK} %显示中文宏包
\usepackage{float}
\usepackage{diagbox}
\usepackage{bm}
\usepackage{ulem} 
\usepackage{amssymb}
\usepackage{soul}
\usepackage{color}
\usepackage{geometry}
\usepackage{fancybox} %花里胡哨的盒子
\usepackage{xhfill} %填充包, 可画分割线 https://www.latexstudio.net/archives/8245
\usepackage{multicol} %多栏包
\usepackage{enumitem}
%\usepackage{enumerate} %可以方便地自定义枚举标题
\usepackage{multirow} %表格中多行单元格合并
\usepackage{wasysym} %可以使用wasysym里的一堆奇奇怪怪的符号
\usepackage{hyperref} % url
%%%%%%%%%%%%%%%伪代码%%%%%%%%%%%%%%%
\usepackage{amsmath}
\usepackage{algorithm}
\usepackage{algorithmicx}
\usepackage[noend]{algpseudocode}
%%%%%%%%%%%%%%%画图包%%%%%%%%%%%%%%%
\usepackage{tikz}
\usepackage{pgfplots} % http://pgfplots.sourceforge.net/gallery.html
\usetikzlibrary{pgfplots.patchplots} % 拟合支持
\usetikzlibrary{arrows,shapes,automata,petri,positioning,calc} % 状态图支持
\usetikzlibrary{arrows.meta} % 箭头
\usetikzlibrary{shadows} % 阴影支持
\usepackage{forest} % 画树

\geometry{left = 1.5cm, right = 1.5cm, top=1.5cm, bottom=2cm}

\definecolor{mygreen}{rgb}{0,0.6,0}
\definecolor{mygray}{rgb}{0.5,0.5,0.5}
\definecolor{mymauve}{rgb}{0.58,0,0.82}
\lstset{
	backgroundcolor=\color{white}, 
	%\tiny < \scriptsize < \footnotesize < \small < \normalsize < \large < \Large < \LARGE < \huge < \Huge
	basicstyle = \footnotesize,       
	breakatwhitespace = false,        
	breaklines = true,                 
	captionpos = b,                    
	commentstyle = \color{mygreen}\bfseries,
	extendedchars = false,
	frame = shadowbox, 
	framerule=0.5pt,
	keepspaces=true,
	keywordstyle=\color{blue}\bfseries, % keyword style
	language = C++,                     % the language of code
	otherkeywords={string}, 
	numbers=left, 
	numbersep=5pt,
	numberstyle=\tiny\color{mygray},
	rulecolor=\color{black},         
	showspaces=false,  
	showstringspaces=false, 
	showtabs=false,    
	stepnumber=1,         
	stringstyle=\color{mymauve},        % string literal style
	tabsize=4,          
	title=\lstname           
}

%\sum\nolimits_{j=1}^{M}   上下标位于求和符号的水平右端,
%\sum\limits_{j=1}^{M}   上下标位于求和符号的上下处,
%\sum_{j=1}^{M}  对上下标位置没有设定,会随公式所处环境自动调整。

%%%%%%%%%%%%%画图包%%%%%%%%%%%%%
\usepackage{tikz}
%%%%%%%%%%%%%好看的矩形%%%%%%%%%%%%%
\tikzset{
  rect1/.style = {
    shape = rectangle,% 指定样式
    minimum height=2cm,% 最小高度
    minimum width=4cm,% 最小宽度
    align = center,% 文字居中
    drop shadow,% 阴影
  }
}
%%%%%%%%%%%%%画图背景包%%%%%%%%%%%%%
\usetikzlibrary{backgrounds}

%%%%%%%%%%%%%在tikz中画一个顶点%%%%%%%%%%%%%
%%%%%%%%%%%%%#1:node名称%%%%%%%%%%%%%
%%%%%%%%%%%%%#2:位置%%%%%%%%%%%%%
%%%%%%%%%%%%%#3:标签%%%%%%%%%%%%%
\newcommand{\newVertex}[3]{\node[circle, draw=black, line width=1pt, scale=0.8] (#1) at #2{#3}}
%%%%%%%%%%%%%在tikz中画一条边%%%%%%%%%%%%%
\newcommand{\newEdge}[2]{\draw [black,very thick](#1)--(#2)}
%%%%%%%%%%%%%在tikz中放一个标签%%%%%%%%%%%%%
%%%%%%%%%%%%%#1:名称%%%%%%%%%%%%%
%%%%%%%%%%%%%#2:位置%%%%%%%%%%%%%
%%%%%%%%%%%%%#3:标签内容%%%%%%%%%%%%%
\newcommand{\newLabel}[3]{\node[line width=1pt] (#1) at #2{#3}}

%%%%%%%%%%%%%强制跳过一行%%%%%%%%%%%%%
\newcommand{\jumpLine} {\hspace*{\fill} \par}
%%%%%%%%%%%%%关键点指令,可用itemise替代%%%%%%%%%%%%%
\newcommand{\keypoint}[2]{$\bullet$\textbf{#1}\quad#2\par}
%%%%%%%%%%%%%<T>平均值表示%%%%%%%%%%%%%
\newcommand{\average}[1]{\left\langle #1\right\rangle }
%%%%%%%%%%%%%表格内嵌套表格%%%%%%%%%%%%%
\newcommand{\tabincell}[2]{\begin{tabular}{@{}#1@{}}#2\end{tabular}}
%%%%%%%%%%%%%大黑点item头%%%%%%%%%%%%%
\newcommand{\itemblt}{\item[$\bullet$]}
%%%%%%%%%%%%%大圈item头%%%%%%%%%%%%%
\newcommand{\itemc}{\item[$\circ$]}
%%%%%%%%%%%%%大星星item头%%%%%%%%%%%%%
\newcommand{\itembs}{\item[$\bigstar$]}
%%%%%%%%%%%%%右▷item头%%%%%%%%%%%%%
\newcommand{\itemrhd}{\item[$\rhd$]}
%%%%%%%%%%%%%定义为%%%%%%%%%%%%%
\newcommand{\defas}{=_{df}}
%%%%%%%%%%%%%偏导%%%%%%%%%%%%%
\newcommand{\partialx}[2]{\frac{\partial #1}{\partial #2}}
%%%%%%%%%%%%%蕴含%%%%%%%%%%%%%
\newcommand{\imp}{\rightarrow}
%%%%%%%%%%%%%上取整%%%%%%%%%%%%%
\newcommand{\ceil}[1]{\lceil#1\rceil}
%%%%%%%%%%%%%下取整%%%%%%%%%%%%%
\newcommand{\floor}[1]{\lfloor#1\rfloor}

%%%%%%%%%%%%%双线分割线%%%%%%%%%%%%%
\newcommand*{\doublerule}{\hrule width \hsize height 1pt \kern 0.5mm \hrule width \hsize height 2pt}
%%%%%%%%%%%%%双线中间可加东西的分割线%%%%%%%%%%%%%
\newcommand\doublerulefill{\leavevmode\leaders\vbox{\hrule width .1pt\kern1pt\hrule}\hfill\kern0pt }
%%%%%%%%%%%%%左大括号%%%%%%%%%%%%%
\newcommand{\leftbig}[1]{\left\{\begin{array}{l}#1\end{array}\right.}
%%%%%%%%%%%%%矩阵%%%%%%%%%%%%%
\newcommand{\mat}[2]{\left[\begin{array}{#1}#2\end{array}\right]}
%%%%%%%%%%%%%可换行圆角文本框%%%%%%%%%%%%%
\newcommand{\ovalboxn}[1]{\ovalbox{\tabincell{l}{#1}}}
%%%%%%%%%%%%%设置section的counter, 使从1开始%%%%%%%%%%%%%
\setcounter{section}{0}

%%%%%%%%%%%%%Colors%%%%%%%%%%%%%
\newcommand{\lightercolor}[3]{% Reference Color, Percentage, New Color Name
    \colorlet{#3}{#1!#2!white}
}
\newcommand{\darkercolor}[3]{% Reference Color, Percentage, New Color Name
    \colorlet{#3}{#1!#2!black}
}
\definecolor{aquamarine}{rgb}{0.5, 1.0, 0.83}
\definecolor{Seashell}{RGB}{255, 245, 238} %背景色浅一点的
\definecolor{Firebrick4}{RGB}{255, 0, 0}%文字颜色红一点的
\lightercolor{gray}{20}{lgray}
\newcommand{\hlg}[1]{
	\begingroup
		\sethlcolor{lgray}%背景色
		\textcolor{black}{\hl{\mbox{#1}}}%textcolor里面对应文字颜色
	\endgroup
}



\title{人工智能基础 HW5}
\author{PB18111697 王章瀚}

\begin{document}
\maketitle
\section*{7.13}
\noindent\textbf{This exercise looks into the relationship between clauses and implication sentences.}
\subsection*{a.}
\noindent \textbf{Show that the clause $(\lnot P_1\lor \cdots\lor \lnot P_m\lor Q)$ is logically equivalent to the implication sentence $(P_1\land\cdots\land P_m)\Rightarrow Q$}\\
\jumpLine\noindent 
Proof.
\begin{align*}
	(P_1\land\cdots\land P_m)\Rightarrow Q &\Leftrightarrow \lnot(P_1\land\cdots\land P_m)\lor Q \\
	&\Leftrightarrow (\lnot P_1\lor\cdots\lor\lnot P_m)\lor Q \\
	&\Leftrightarrow \lnot P_1\lor\cdots\lor\lnot P_m\lor Q
\end{align*}
\hfill $\square$

\subsection*{b.}
\noindent \textbf{Show that every clause (regardless of the number of positive literals) can be written in the form $(P_1\land\cdots\land P_m)\Rightarrow (Q_1\lor\cdots\lor Q_n)$, where the $P_s$ and $Q_s$ are proposition symbols. A knowledge base consisting of such sentences is in implicative normal form or Kowalski form (Kowalski, 1979). }\\
\jumpLine\noindent 
I first show that $(P_1\land\cdots\land P_m)\Rightarrow (Q_1\lor\cdots\lor Q_n)$ is equivalent to $\lnot P_1\lor\cdots\lor\lnot P_m\lor Q_1\lor\cdots\lor Q_n$:
\begin{align*}
	(P_1\land\cdots\land P_m)\Rightarrow (Q_1\lor\cdots\lor Q_n) &\Leftrightarrow \lnot(P_1\land\cdots\land P_m)\lor (Q_1\lor\cdots\lor Q_n) \\
	&\Leftrightarrow (\lnot P_1\lor\cdots\lor\lnot P_m)\lor (Q_1\lor\cdots\lor Q_n) \\
	&\Leftrightarrow \lnot P_1\lor\cdots\lor\lnot P_m\lor Q_1\lor\cdots\lor Q_n
\end{align*}
Secondly, each clause can be written in the form of $\lnot P_1\lor\cdots\lor\lnot P_m\lor Q_1\lor\cdots\lor Q_n$. \\
Thus, every clause can be written in the form of $(P_1\land\cdots\land P_m)\Rightarrow (Q_1\lor\cdots\lor Q_n)$

\subsection*{c.}
\noindent \textbf{Write down the full resolution rule for sentences in implicative normal form.}\\
\jumpLine\noindent 
The resolution rule is:
$$\frac{l_1\lor\cdots\lor l_k,\qquad m_1\lor\cdots\lor m_n}{l_1\lor\cdots\lor l_{i-1}\lor l_{i+1}\lor\cdots\lor l_k\lor m_1\lor\cdots\lor m_{j-1}\lor m_{j+1} \lor\cdots\lor m_n}$$
where $l_i$ and $m_j$ are complementary literals.\\
This can be rewritten as:
$$\frac{\lnot P_1\lor\cdots\lor\lnot P_m\lor Q_1\lor\cdots\lor Q_n,\qquad \lnot R_1\lor\cdots\lor\lnot R_k\lor S_1\lor\cdots\lor S_l}{P_1\lor\cdots\lor P_{i-1}\lor P_{i+1}\lor\cdots\lor P_m\lor Q_1\lor\cdots\lor Q_n\lor \lnot R_1\lor\cdots\lor\lnot R_k S_1\lor\cdots\lor S_{j-1}\lor S_{j+1} \lor\cdots\lor S_l}$$
where $P_i=S_j$. According to (b), the rule above can be written as:
$$\frac{(P_1\land\cdots\land P_m)\Rightarrow (Q_1\lor\cdots\lor Q_n),\qquad (R_1\land\cdots\land R_k)\Rightarrow (S_1\lor\cdots\lor S_l)}{P_1\lor\cdots\lor P_{i-1}\lor P_{i+1}\lor\cdots\lor P_m\lor Q_1\lor\cdots\lor Q_n\lor \lnot R_1\lor\cdots\lor\lnot R_k\lor  S_1\lor\cdots\lor S_{j-1}\lor S_{j+1} \lor\cdots\lor S_l}$$



\section*{Proof}
\noindent \textbf{Prove the completeness of the forward chaining algorithm.}\\
\jumpLine\noindent 
\begin{enumerate}[label=\arabic*. ]
	\item 当 FC 到达不动点后, 不会再有新的推理
	\item 考虑算法伪代码中的 inferred 表的最终状态, 参与推理过程的均为 true, 否则为 false. 可以把这个推理表看成一个模型 M, 
	\item 且原始 KB 中的每个子句在模型 M 中都为真. 其证明如下:
	\begin{enumerate}[label=(\arabic*). ]
		\item 如果有某个子句 $a_1\land\cdots\land a_k\Rightarrow b$ 在 M 中为 false, 那么 $a_1\land\cdots\land a_k$ 在 M 中为 true 且 $b$ 在 M 中为 false
		\item 但由于算法已经到达了不动点, 故此条知识应得以继续推理, 矛盾.
	\end{enumerate}
	\item 因此 M 是 KB 的一个模型. 若 $KB\vDash q$, 则 $q$ 在 KB 的所有模型中为 true, 也即在 M 下为 true.
	\item 因此 $q$ 在 inferred 表中也为 true, 进而能够被 FC 算法推断出来.
\end{enumerate}

\end{document}




