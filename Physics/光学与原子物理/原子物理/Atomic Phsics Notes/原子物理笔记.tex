\documentclass[UTF8]{article}
\usepackage{amsmath}
\usepackage{makecell}
\usepackage[utf8]{inputenc}
\usepackage[space]{ctex} %中文包
\usepackage{listings} %放代码
\usepackage{xcolor} %代码着色宏包
\usepackage{CJK} %显示中文宏包
\usepackage{bm}
\usepackage{ulem} 
\usepackage{amssymb}
\usepackage{soul}
\usepackage{color}
\usepackage{float}
\usepackage{diagbox}
\usepackage{makecell}

\definecolor{mygreen}{rgb}{0,0.6,0}
\definecolor{mygray}{rgb}{0.5,0.5,0.5}
\definecolor{mymauve}{rgb}{0.58,0,0.82}
\lstset{
	backgroundcolor=\color{white}, 
	basicstyle = \footnotesize,       
	breakatwhitespace = false,        
	breaklines = true,                 
	captionpos = b,                    
	commentstyle = \color{mygreen}\bfseries,
	extendedchars = false,
	frame =shadowbox, 
	framerule=0.5pt,
	keepspaces=true,
	keywordstyle=\color{blue}\bfseries, % keyword style
	language = C++,                     % the language of code
	otherkeywords={string}, 
	numbers=left, 
	numbersep=5pt,
	numberstyle=\tiny\color{mygray},
	rulecolor=\color{black},         
	showspaces=false,  
	showstringspaces=false, 
	showtabs=false,    
	stepnumber=1,         
	stringstyle=\color{mymauve},        % string literal style
	tabsize=4,          
	title=\lstname                      
}


%画图包
\usepackage{tikz}
%画图背景包
\usetikzlibrary{backgrounds}

%自定义命令
\newcommand{\psiG}{\psi_{G}}
%在tikz中画一个顶点
%#1:node名称
%#2:位置
%#3:标签
\newcommand{\newVertex}[3]{\node[circle, draw=black, line width=1pt, scale=0.8] (#1) at #2{#3}}
%在tikz中画一条边
\newcommand{\newEdge}[2]{\draw [black,very thick](#1)--(#2)}
%在tikz中放一个标签
%#1:名称
%#2:位置
%#3:标签内容
\newcommand{\newLabel}[3]{\node[line width=1pt] (#1) at #2{#3}}
\newcommand{\keypoint}[2]{$\bullet$\textbf{#1}\quad#2\par}
\newcommand{\vr}{\bm{r}}
\newcommand{\average}[1]{\left\langle #1\right\rangle }
\newcommand{\jumpline} {\hspace*{\fill} \par}
\newcommand{\tabincell}[2]{\begin{tabular}{@{}#1@{}}#2\end{tabular}} %用于表格中换行

\begin{document}

\section{玻尔原子模型}
\subsection{原子的核式结构}
\subsubsection{电子的发现}
\keypoint{汤姆孙测电子荷质比}{$\frac{e}{m}=\frac{Eh}{LlB^2}$}
\subsubsection{$\bm{\alpha}$粒子散射实验}
\keypoint{$\alpha$粒子}{氦核,带两个正电荷}
\keypoint{偏转角}{$\theta\approx\frac{F\Delta t}{p}=\frac{2Ze^2}{4\pi\epsilon_0Rmv^2}=\frac{Ze^2}{4\pi\epsilon_0RE}$, 其中$F=\frac{2Ze^2r}{4\pi\epsilon_0R^3}\ (r<R)$}
\subsubsection{卢瑟福核式模型}
\textcolor{blue}{\keypoint{散射公式}{
	$$ 
	\left\{
	\begin{array}{l}
	cot\frac{\theta}{2}=4\pi\epsilon_0\frac{mv_0^2}{zZe^2}b=\frac{2b}{D},\ b\mbox{是瞄准距离}\\
	D=\left(\frac{1}{4\pi\epsilon_0}\right)\frac{2zZe^2}{mv_0^2}=\left(\frac{zZe^2}{4\pi\epsilon_0}\right)\left(\frac{1}{E}\right),\ \mbox{库仑散射因子}\\
	\end{array}
	\right.
	$$
}}
\textcolor{blue}{\keypoint{原子微分散射截面公式}{
	$$d\Omega=2\pi sin\theta d\theta$$
	$$d\sigma=\frac{D^2}{16}\frac{1}{sin^4\frac{\theta}{2}}d\Omega$$
	其中$d\sigma$代表入射粒子被一个靶原子散射到$\theta$方向的$d\Omega$内概率, $\frac{d\sigma(\theta)}{d\Omega}$表示散射到$\theta$方向上单位立体角内的概率.
}}
\keypoint{如果射在薄靶上}{
	靶厚度为$t$,面积为$A$,单位体积含有原子数$N$(原子数/$cm^3$),\par
	\textcolor{blue}{原子的有效面积: 
	$$dA=2\pi b(db)NtA=(d\sigma)NtA$$}\par
	入射粒子在有效面积内概率: 
	$$\frac{dA}{A}=2\pi bNt(db)=Nt(d\sigma)$$\par
	入射粒子数为$n$,散射到$\theta$方向上的为$dn$,则散射到$\theta$方向概率为: 
	$$\frac{dn}{n}=\frac{dA}{A}=Ntd\sigma$$\par
	粒子散射在单位立体角内概率:
	$$\frac{dn}{nd\Omega}=\frac{d\sigma}{d\Omega}\cdot Nt=\frac{D^2}{16}\frac{1}{sin^4\frac{\theta}{2}}\cdot Nt$$\par
	从而:
	$$\frac{dn}{nd\Omega}=\left(\frac{1}{4\pi\epsilon_0}\right)^2\left(\frac{zZe^2}{4E}\right)^2\cdot Nt\cdot \frac{1}{sin^4\frac{\theta}{2}}$$\par
}
\textcolor{blue}{\keypoint{卢瑟福散射公式得到最小距离,给出原子核大小上限}{$r_m=\frac{D}{2}\left(1+\frac{1}{sin\frac{\theta}{2}}\right)$}}

\subsection{原子光谱}
\subsubsection{光谱}
\subsubsection{氢原子光谱和光谱项}
\keypoint{巴耳末公式}{$\lambda_n=B\frac{n^2}{n^2-4},\qquad n=3,4,5,\cdots$}
\keypoint{光谱波数$\tilde{v}$公式}{
	\begin{align*}
	\tilde{v}&=R_H\left(\frac{1}{m^2}-\frac{1}{n^2}\right),\ m=1,2,3,\cdots;n=m+1,m+2,\cdots\\
	&=T_m-T_n
	\end{align*}
	其中,$R_H=\frac{4}{B}$是里德伯常量;$T_n=\frac{R}{n^2}$称为谱项.巴耳末系相当于$n=2$的情况.\par
}
\subsection{玻尔氢原子理论}
\subsubsection{原子行星模型的困难}
\keypoint{行星模型下的一些分析}{
	\par
	电子圆周运动的向心力:
	$$\frac{m_ev^2}{r}=\frac{1}{4\pi\epsilon_0}\frac{Ze^2}{r^2}$$\par
	总能量:
	$$E=\frac{1}{2}m_ev^2-\frac{1}{4\pi\epsilon_0}\frac{Ze^2}{r}=-\frac{1}{2}\frac{1}{4\pi\epsilon_0}\frac{Ze^2}{r}$$\par
	电子运动频率(不应如此,否则光谱连续):
	$$f=\frac{v}{2\pi r}=\frac{e}{2\pi}\sqrt{\frac{Z}{4\pi\epsilon_0m_er^3}}$$\par
	
}
\subsubsection{玻尔假设}
\keypoint{玻尔假设}{
	$$
	\left\{
	\begin{array}{l}
	\mbox{定态不发射}\\
	\mbox{发射吸收电磁辐射频率: }\nu=\frac{|E_n-E_m|}{h}\\
	\mbox{角动量量子化: }mrv=n\hbar,\qquad n=1,2,3,\cdots\\
	\end{array}
	\right.
	$$
}
\subsubsection{玻尔氢原子模型}
\textcolor{blue}{\keypoint{玻尔氢原子模型(三个假设结合行星模型)}{
	$$
	\left\{
	\begin{array}{l}
	v_n=\frac{e^2}{4\pi\epsilon_0\hbar}\times\frac{Z}{n}\\
	r_n=\frac{4\pi\epsilon_0\hbar^2}{m_ee^2}\times\frac{n^2}{Z}\\
	\end{array}
	\right.
	$$
}}
\textcolor{blue}{\keypoint{玻尔半径}{上式$r_n$中$n=1,Z=1$时,有玻尔半径: \colorbox{yellow}{$a_0=\frac{4\pi\epsilon_0\hbar^2}{m_ee^2}$}}}
\textcolor{blue}{\keypoint{精细结构常数}{\colorbox{yellow}{$\alpha=\frac{e^2}{4\pi\epsilon_0\hbar c}$},则$v_n=\frac{\alpha cZ}{n},$}}
\textcolor{blue}{\keypoint{氢原子的定态能量}{$$E_n=-\frac{2\pi^2m_ee^4}{(4\pi\epsilon_0)^2h^2n^2}=-\frac{1}{2n^2}m_e\alpha^2c^2$$ 其中n=1时得到氢原子基态能量$E_1=-\frac{1}{2}m_ec^2\alpha^2\approx-13.6ev$,约为电子静质量能的$\alpha^2$倍}}
\keypoint{氢原子光谱}{
	波数(由\textcolor{blue}{$hc\tilde{\nu}=E_n-E_m$}):
	$$\tilde{\nu}=\frac{\alpha^2m_ec}{2h}\left(\frac{1}{m^2}-\frac{1}{n^2}\right)$$
	其中可得原子核质量无穷大的里德伯常量:
	\textcolor{blue}{$$R_\infty=\frac{\alpha^2m_ec}{2h}$$}
}
\subsection{类氢原子}
\subsubsection{原子核质量的影响}
\textcolor{blue}{\keypoint{约化质量}{$\mu=\frac{Mm_e}{M+m_e}$}
\keypoint{约化质量的使用}{将上一节的所有$m_e$换成$\mu$即可.}}
\keypoint{轨道半径}{$r_n=\frac{4\pi\epsilon_0\hbar^2}{\mu e^2}\times\frac{n^2}{Z}$}
\keypoint{原子定态能量}{$E_n=-\frac{2\pi^2\mu e^4}{(4\pi\epsilon_0)^2h^2n^2}=-\frac{1}{2n^2}\mu\alpha^2c^2$}
\keypoint{里德伯常量}{$R_m=R_\infty\frac{\mu}{m_e}=\frac{\alpha^2\mu c}{2h}$}
\subsubsection{类氢离子光谱}
\keypoint{毕克林线系}{
	对于类氢离子,在相邻两条巴耳末线之间,满足
	$$\tilde{\nu}=R\left(\frac{1}{2^2}-\frac{1}{k^2}\right),\ k=\frac{5}{2},3,\frac{7}{2},\cdots$$
	这与巴耳末线系公式相同,但多了半整数.
	这是由于对于类氢离子,要考虑$Z$.而$E_n=-\frac{1}{2}\mu v_n^2$,$v_n$中有个$Z$,因此$E_n\propto Z^2$,所以应该是
	$$\tilde{\nu}=4R\left(\frac{1}{m^2}-\frac{1}{n^2}\right),\ m=1,2,3,\cdots;n=m+1,m+2,\cdots$$
}
\subsubsection{特殊的类氢离子}
\keypoint{$\bm{\mu^-}$原子}{就像是很重的电子.}
\keypoint{里德伯原子}{原子中电子被激发到很高能级状态时,激发电子的轨道远离其他电子,内部就是由原子核和所有的内电子组成的一个电荷为$+e$的原子实,这种原子成为里德伯原子}
\subsection{弗兰克-赫兹实验}
\textbf{除了光谱学方法外的另一种证明能级分立的方法.}\par
\keypoint{弗兰克-赫兹实验基本思想}{利用加速电子碰撞原子,使之激发。测量电子所损失的能量,即是原子所吸收的能量。
}
\keypoint{主要过程}{$V_G$足够大使原子激发到第一激发态时,随着$V_G$增加,电子动能增加;直到原子能被激发到第二能级,则电流骤降.再以此类推.}
\rule{\textwidth}{0.50mm}



\section{量子力学的初步介绍}
\subsection{波粒二象性}
\subsubsection{光的二象性}
\keypoint{辐射强度$I$}{题3.8有个: $I\propto \lambda_{if}\cdot N_k\cdot E_{if}$}
\keypoint{瑞利金斯公式}{$u(\nu,T)=\frac{8\pi}{c^3}kT\nu^2$, 问题在于当$\nu\rightarrow\infty$,紫外灾难.}
\keypoint{普朗克黑体辐射公式}{$u(\nu, T)=\frac{8\pi}{c^{3}}\frac{h\nu^{3}}{e^{h\nu/kT}-1}$}
\keypoint{光电效应公式}{$\frac{1}{2}m_{e}v^2=h\nu-W$}
\textcolor{blue}{\keypoint{光子动量}{$p=\frac{h\nu}{c}=\frac{h}{\lambda}$}}
\textcolor{blue}{\keypoint{光子动能}{$E=h\nu$}}
\keypoint{相对论静止质量和能量的普遍关系}{$(m_cc)^2=E^2-(cp)^2$}
\textcolor{blue}{\keypoint{康普顿效应公式}{$\Delta\lambda=\lambda'-\lambda=\frac{h}{mc}(1-cos\theta)$,其中$m$是靶物质质量(题目中可能会说“光子被一个啥玩意散射”),$\lambda $和$\lambda'$是入、散射光子波长,$\theta$是出射方向}}
\keypoint{康普顿波长}{$\frac{h}{m_{e}c}$}
\subsubsection{实物粒子的波动性}
\textcolor{blue}{\keypoint{非相对论下实物粒子波长}{$\lambda\approx\frac{h}{\sqrt{2m_{0}E_{k}}}=\frac{h}{m_{0}v}=\frac{h}{p}$}}
\subsection{物质波的统计解释和海森伯不确定原理}
\subsubsection{波函数的统计解释}
\keypoint{平面单色波}{$\Psi(\bm{r},t)=\Psi_{0}cos(\bm{k\cdot r}-\omega t),\ where\  \bm{k}=\frac{2\pi\bm{p}}{h}$}
\keypoint{复数表示}{$\Psi(\bm{r},t)=\Psi_{0}exp[i(\bm{k\cdot r}-\omega t)]$}
\keypoint{用粒子能量动量为参数表示}{$\Psi(\bm{r},t)=\Psi_{0}exp[\frac{i}{\hbar}(\bm{p\cdot r}-Et)]$}
\textcolor{blue}{\keypoint{海森伯不确定原理}{若$p,q$不对易($pq-qp\neq 0$),则$\Delta p\cdot\Delta q\ge \hbar/2$}
\keypoint{位置和动量不确定关系}{$\Delta p_{x}\cdot\Delta x\ge \hbar/2$, (the same for y,z)}
\keypoint{能量和时间不确定关系}{$\Delta E\cdot\Delta t\ge \hbar/2$}}
\textcolor{blue}{\keypoint{能级宽度}{$\Gamma\approx\frac{\hbar}{\tau}$}}
\subsection{薛定谔方程}
\keypoint{薛定谔波动方程}{$i\hbar\frac{\partial}{\partial t}\Psi=-\frac{\hbar^{2}}{2m}\nabla^{2}\Psi+V(\bm{r},t)\Psi$}
\keypoint{定态薛定谔方程}{
	如果势场V不显含t,就可以分离变量.此时可以写成$\Psi(\bm{r},t)=u(\bm{r})f(t)$.带入后得到(结合左边只和t有关,右边只和$\bm{r}$有关)\par
	\begin{align*}
	E\triangleq \frac{i\hbar}{f(t)}\frac{df}{dt}&=\frac{1}{u(\bm{r})}[-\frac{\hbar^{2}}{2m}\nabla^{2}u(\bm{r})+V(\bm{r})u(\bm{r})]
	\end{align*}\par
	解常微分方程$E=\frac{i\hbar}{f(t)}\frac{df}{dt}$得$f(t)=f(0)exp(\frac{-iEt}{\hbar})$\par
	将常数$f(0)$并入$u(\bm{r})$得$\Psi(\bm{r},t)=u(\bm{r})exp(\frac{-iEt}{\hbar})$\par
	\textcolor{blue}{其中$u(\bm{r})$满足\textbf{定态薛定谔方程}(不含时间量)
	$$[-\frac{\hbar^{2}}{2m}\nabla^{2}+V(\bm{r})]u(\bm{r})=Eu(\bm{r})$$}
	其实此处引入的$E$就是粒子总能量值.\par
}
\subsection{力学量的平均值、算符表示和本征值}
\subsubsection{力学量的平均值}
\keypoint{平均位置}{$\average{\vr} =\int_{-\infty}^{\infty}\vr P(\vr,t)d\vr $}
\keypoint{平均势能}{$\average{V(\vr,t)} =\int_{-\infty}^{\infty}V(\vr,t) P(\vr,t)d\vr $}
\keypoint{动量算符}{$\hat{p}=-i\hbar\nabla$}
\keypoint{平均动量}{$\average{p_{x}} = \int_{-\infty}^{\infty}\Psi^{*}(\vr,t)\hat{p}\Psi(\vr,t)d\vr $}
\keypoint{动能算符}{$\hat{T}=-\frac{\hbar^{2}}{2m}\nabla^{2}$}
\keypoint{平均动能}{$\average{T} = \int_{-\infty}^{\infty}\Psi^{*}(\vr,t)\hat{T}\Psi(\vr,t)d\vr $}
\keypoint{能量算符}{哈密顿量$H=T+V$,$\hat{H}=-\frac{\hbar^{2}}{2m}\nabla^{2}+V(\vr)$}
\keypoint{平均能量}{$\average{E} = \int_{-\infty}^{\infty}\Psi^{*}(\vr,t)\hat{H}\Psi(\vr,t)d\vr =\int_{-\infty}^{\infty}\Psi^{*}(\vr,t)\left( i\hbar\frac{\partial}{\partial t}\right) \Psi(\vr,t)d\vr$}
\keypoint{角动量算符}{$\hat{L}_x=y\hat{p}_z-z\hat{p}_y=-i\hbar\left(y\frac{\partial}{\partial z}-z\frac{\partial}{\partial y}\right)$, $y-x,z;z-y,z$.}
\subsubsection{力学量的本征值}
\textcolor{blue}{\keypoint{本征方程}{
	$$\hat{H}u(\bm{r})=Eu(\bm{r})$$
	$E$称为算符$\hat{H}$的\textbf{本征值}(可以看作是期望值),$u(\bm{r})$称为算符$\hat{H}$的\textbf{本征函数}.本征函数表示的态称为\textbf{本征态}.
}}	
\keypoint{算符对易}{波函数$\psi$同时是算符$\hat{\Omega}^{(1)}$和$\hat{\Omega}^{(2)}$的本征函数,则两个物理量可以同时有确定值,此时$\hat{\Omega}^{(1)}\hat{\Omega}^{(2)}-\hat{\Omega}^{(2)}\hat{\Omega}^{(1)}=0$,记作$[\hat{\Omega}^{(1)},\hat{\Omega}^{(2)}]=0$}
\subsection{定态薛定谔方程解的几个简例}
就是解$\hat{H}\Psi=E\Psi$,即$-\frac{\hbar^{2}}{2m}\frac{d^{2}ux)}{dx{2}}+V(x)u(x)=Eu(x)$\par
基本过程是:列出不同V处的方程,解微分方程.再由单值,有限,连续等条件得到确定最终方程.最后可以再一步归一化.\par
\textcolor{blue}{\keypoint{透入距离}{$\Delta x\approx\frac{1}{k_2}=\frac{\hbar}{\sqrt{2m(V_0-E)}}$}} 
\rule{\textwidth}{0.50mm}



\section{单电子原子}
\subsection{氢原子的定态薛定谔方程解}
\subsubsection{中心力场薛定谔方程及其解}
\keypoint{电子静电势能}{$V(r)=-\frac{e^{2}}{4\pi \epsilon_{0}r}$}
\keypoint{球坐标拉普拉斯算符}{$\nabla^{2}=\frac{1}{r^{2}}\frac{\partial}{\partial r}\left(r^{2}\frac{\partial}{\partial r}\right)+\frac{1}{r^{2}sin\theta}\frac{\partial}{\partial \theta}\left(sin\theta \frac{\partial}{\partial \theta}\right)+\frac{1}{r^{2}sin\theta}\frac{\partial^{2}}{\partial \varphi^{2}}$}
\keypoint{三个微分方程}{
	$$ \left\{
	\begin{array}{l}
	\frac{d^2\Phi}{d\varphi^{2}}+m_{l}^{2}\Phi=0\\
	-\frac{1}{sin\theta}\frac{d}{d\theta}\left(sin\theta\frac{d\Theta}{d\theta}\right)+\frac{m_{l}^{2}}{sin^{2}\theta}\Theta=l(l+1)\Theta\\
	\frac{d}{dr}\left(r^{2}\frac{dR}{dr}\right)+\frac{2mr^{2}}{\hbar^{2}}\left(E+\frac{e^{2}}{4\pi \epsilon_{0}r}\right)R=l(l+1)R\\
	\end{array} 
	\right .$$
}
\textcolor{blue}{\keypoint{三个微分方程的解}{
	$$ \left\{
	\begin{array}{lr}
	\Phi_{m_{l}}(\varphi)=Ae^{im_{l}\varphi}\qquad,\mbox{从指数看出$m_{l}$}\\
	\Theta_{lm}=BP_{l}^{m}(cos\theta)\qquad,\mbox{从三角函数总次数看出$l$}\\
	R_{n,l}=C_{n,l}exp\left(-\frac{r}{na_{0}}\right)\left(\frac{2r}{na_{0}}\right)^{l}L_{n+l}^{2l+1}\left(\frac{2r}{na_{0}}\right),\ \colorbox{yellow}{$a_{0}=\frac{4\pi\epsilon_{0}\hbar^{2}}{me^{2}}$}\qquad,\mbox{从指数的分母看出$n$}\\
	\mbox{其中有:}
		\left\{
		\begin{array}{l}
		n=1,2,3,4,\cdots;\\
		l=0,1,2,\cdots,(n-1)\\
		m_{l}=-l,-l+1,\cdots,l-1,l\\
		\end{array} 
		\right\}
	\end{array} 
	\right .$$
}}
\subsubsection{概率密度}
\keypoint{电子分布概率}{
	$$ \left\{
	\begin{array}{l}
	P(\varphi)d\varphi=\frac{1}{2\pi}d\varphi\\
	P(\theta)d\theta=\Theta_{lm}^{*}\Theta_{lm}sin\theta d\theta\\
	P(r)dr=R_{nl}^{*}R_{nl}r^{2}dr\\
	\end{array} 
	\right.$$
}
\keypoint{概率密度最大值}{对$l=n-1$态,概率密度最大值在$r_m=n^2a_0$处,和玻尔理论的轨道半径相同.原子中电子在玻尔理论的电子轨道处出现的概率最大(最概然半径).}
\keypoint{量子力学电子半径期望值}{为$1.5a_0$,不同于上述最概然半径$a_0$.}
\subsubsection{原子波函数的宇称}
\subsection{量子数的物理解释}
\subsubsection{主量子数$n$、单电子原子的能级}
\keypoint{束缚电子能量本征值}{$E_{n}=-\alpha^{2}\frac{mc^{2}Z^{2}}{2n^{2}}$,其中$\alpha=\frac{e^{2}}{4\pi\epsilon_{0}\hbar c}$.本式其实仍可由$E_n=\frac{1}{2}mv_n^2$得到,只不过$v_n=\frac{\alpha cZ}{n}$.}
\keypoint{主量子数$n$}{主量子数$n$给定的是原子的总能量(能量确定了,发生了\textbf{简并}),但波函数还要看$l$和$m_{l}$}
\keypoint{简并}{单电子势场是库仑势场,从而得到了$l$的简并性.而无外场时总能量与原子的轨道角动量取向无关,故有$m_l$的简并性.此外,对于同一个n,共有$\sum\limits_{l=0}^{n-1}(2l+1)=n^2$重简并.}
\keypoint{对量子数$l$的标记}{$l=0,1,2,3,4,,\cdots\mbox{对应}s,p,d,f,g,\cdots$}
\subsubsection{轨道角动量及量子数$l$}
\keypoint{轨道角动量}{$\bm{\hat{l}}=\frac{\hbar}{i}\bm{r}\times\nabla$}
\keypoint{角动量平方算符本征方程}{$\hat{l}u(r,\theta,\varphi)=l(l+1)\hbar^{2}u(r,\theta,\varphi)$}
\textcolor{blue}{\keypoint{轨道角动量大小}{$\left|\bm{l}\right|=\sqrt{l(l+1)}\hbar$,值得注意的是它与玻尔理论给出的$l=n\hbar$不同,而量子力学式才是对的.}}
\keypoint{角动量量子数$l$}{量子态为$(n,l,m_{l})$的单电子原子的轨道角动量之依赖于角动量量子数$l$}
\subsubsection{磁量子数$m_{l}$}
\keypoint{角动量$z$分量算符}{$\hat{l}_{z}=-i\hbar\frac{\partial}{\partial \varphi}$}
\keypoint{角动量$z$分量本征方程}{$\hat{l}_{z}\Phi_{m_{l}}(\varphi)=m_{l}\hbar\Phi_{m_{l}}(\varphi)$}
\textcolor{blue}{\keypoint{$l_{z}$本征值}{$l_{z}=m_l\hbar$}}
\subsubsection{角动量的矢量模型}
\keypoint{$l_x$和$l_y$平均值为0}{$\left\langle l_x\right\rangle =\left\langle l_y\right\rangle =0 $,他们在量子态下没有确定值,轨道角动量守恒,但没有完全确定的方向}
\subsection{跃迁概率和选择定则}
\subsubsection{原子处在定态时不发射电磁辐射}
稳定的电荷分布体系不会发射电磁辐射.\par
\subsubsection{原子跃迁和混合态}
%
% TO BE FILLED
%
\textcolor{red}{这个部分看不懂...以后再搞}\par
\textcolor{red}{以后还是看不懂......}\par
\subsubsection{跃迁率、平均寿命}
\keypoint{电偶极子单位时间辐射的平均能量}{$P=\frac{4\pi^3\nu^4}{3\epsilon_{0}c^3}\left|\bm{p}_0\right|^2$}
\keypoint{电偶极原子单位时间内发生非极化辐射概率}{$\lambda_{if}=\frac{P}{h\nu}=\frac{4\pi^3\nu^3}{3\epsilon_{0}hc^3}\left|\bm{p}_0\right|^2$,从初态$i$到终态$f$}
\keypoint{$dt$时间内从初态跃迁到终态的原子数$dN_{if}$与跃迁率、初态原子数$N_i$及时间间隔成正比}{$dN_{if}=\lambda_{if}N_idt$,结合$dN_{if}=-dN_i$从而积分得:{$N_i(t)=N_i(0)e^{-\lambda_{if}t}$}}
\textcolor{blue}{\keypoint{对任意电荷分布}{$\lambda_{if}=\frac{16\pi^3\nu^3}{3\epsilon_0hc^3}|\bm{p}_{if}|^2$}}
\textcolor{blue}{\keypoint{初态原子平均寿命}{$\tau=\frac{1}{N_i(0)}\int_{0}^{\infty}tdN_i=\frac{1}{\lambda_{if}}$}}
\textcolor{blue}{\keypoint{能量宽度/能级宽度}{$\Gamma=\Delta E=\hbar\Delta\omega$}}
\textcolor{blue}{\keypoint{平均寿命来表示时间不确定度}{$\tau\approx\frac{\hbar}{\Gamma}$}}
\keypoint{\textbf{选择定则}}{
	初末态的量子数满足下列关系时,电偶极跃迁概率才不为0.
	$$ \left\{
	\begin{array}{l}
	\Delta m = m-m'=0,\pm1\\
	\Delta l = l-l'=\pm1\\
	\end{array}
	\right .$$
}

\subsection{电子自旋}
\subsubsection{轨道磁矩}
\keypoint{玻尔磁子$\mu_{B}$(用作单位)}{$\mu_{B}=\frac{e\hbar}{2m_e}$}
\keypoint{轨道磁矩}{$\bm{\mu}_{l}=-\frac{g_le}{2m_e}\bm{l}=-\frac{g_l\mu_B}{\hbar}\bm{l}$}
\keypoint{轨道磁矩大小}{$\mu_{l}=g_l\frac{e\hbar}{2m_e}\sqrt{l(l+1)}$}
\keypoint{磁矩$z$分量}{$\mu_z=-m_lg_l\mu_B$}
\keypoint{拉莫尔频率}{}
\subsubsection{塞曼效应}
\textbf{磁量子数$m_l$导致的能级分裂.}\par
\keypoint{塞曼效应}{光原在外磁场中,原子所发射光谱线会分裂成几条分支谱线,且分裂后各条谱线是偏振的.}
\keypoint{磁矩在磁场中势能}{$\Delta E=-\bm{\mu_l\cdot B}$}
\textcolor{blue}{\keypoint{磁矩在磁场中势能}{$\Delta E_m=\left(\frac{\mu_B}{\hbar}\right)g_j\bm{j\cdot B}=m_jg_j\mu_BB$}}
\keypoint{正常塞曼效应产生谱线频率差}{$h(\nu'-\nu)=\Delta m\mu_B B$,其中,$\Delta m=0,\pm1$}
\keypoint{反常塞曼效应}{似乎是由于总角动量的两种$j$导致的}
\subsubsection{施特恩-格拉赫实验}
\textbf{电子自旋$s$导致的能级分裂.如果没有,则应该只有$2l+1$条斑痕,但有偶数条.}\par
\textcolor{blue}{\keypoint{磁矩在不均匀磁场中受一平移力}{$F=\nabla (\bm{\mu\cdot B})$}
\keypoint{若磁场只在$z$方向不均}{$F_z=\mu_z\frac{dB}{dz}=-m_lg_l\mu_B\frac{dB}{dz}$}
\keypoint{考虑总角动量后}{$F_z=-m_jg_j\mu_B\frac{dB}{dz}$}}
\subsubsection{电子自旋}
电子具有固有角动量,称为\textbf{自旋角动量},简称\textbf{自旋}\par
则,电子具有\textbf{固有磁矩},即\textbf{自旋磁矩};这样电子在原子内部的磁场中将具有一定\textbf{取向势能}.\par
用自旋量子数$s$和自旋磁量子数$m_s$来描述自旋角动量.\par
\textcolor{blue}{\keypoint{自旋角动量及其z分量}{
	$$ \left\{
	\begin{array}{l}
	\bm{s}^2=s(s+1)\hbar^2,\ s=1/2\\
	s_z=m_s\hbar,\ m_s=\pm1/2
	\end{array}
	\right .$$
}}
\textcolor{blue}{\keypoint{自旋磁矩}{$\mu_s=-\frac{g_s\mu_B\bm{s}}{\hbar},\ g_s=2$,这个和轨道磁矩的形式一样$\mu_l=-\frac{g_l\mu_B\bm{l}}{\hbar},\ g_l=1$}
\keypoint{自旋磁矩$z$分量}{$\mu_z=\mp\frac{g_l\mu_B\bm{l}}{\hbar}\frac{\hbar}{2}=\mp\mu_B=\mp\frac{e\hbar}{2m}$}}
\subsection{自旋和轨道相互作用}
\textcolor{blue}{\keypoint{电子因轨道运动感受到一磁场}{$\bm{B}=\frac{1}{2}\frac{1}{m_ec^2}\frac{1}{4\pi\epsilon_{0}}\frac{Ze}{r^3}\bm{l}$,可见受磁场$\bm{B}$与$\bm{l}$同向}}
\subsubsection{自旋-轨道耦合能}
\keypoint{取向势能}{$W=-\bm{\mu}_s\cdot \bm{B}$}
\textcolor{blue}{\keypoint{上述势能使电子能量有增量}{$\Delta E=W=-\bm{\mu}_s\cdot \bm{B}=\frac{1}{4\pi\epsilon_{0}}\frac{Ze^2}{2m_e^2c^2r^3}\bm{s\cdot l}$,其中,自旋量子数$s=1/2$,$\bm{s}$只能有两个取向,$\bm{s\cdot l}$可以有两个值,对应于能级分裂为两层结构.而对于$l=0$的,能级不分裂}}
\subsubsection{总角动量和原子磁矩}
自旋-轨道相互作用使$\bm{l}$和$\bm{s}$的取向彼此相关,轨道角动量和自旋角动量不再守恒.\par
\textcolor{blue}{\keypoint{总角动量}{$\bm{j}=\bm{s}+\bm{l}$}}
\keypoint{总角动量大小}{$\bm{j}^2=\bm{s}^2+\bm{l}^2+2\bm{s\cdot l}$}
\keypoint{总角动量量子数表示}{
	$$ \left\{
	\begin{array}{l}
	\bm{j}^2=j(j+1)\hbar^2\\
	j_z=m_j\hbar\\
	\end{array}
	\right .$$
}
\textcolor{blue}{\keypoint{总角动量量子数关系}{
	$$ \left\{
	\begin{array}{l}
	m_j=-j,\cdots,j\\
	j = l+s,l+s-1,\cdots,\left|l-s\right|\mbox{其实只有}\colorbox{yellow}{$l+1/2,\left|l-1/2\right|$}\\
	\end{array}
	\right .$$
}}
\keypoint{好量子数}{可以表征电子状态.例如此处,$(n, l, j, m_j)$就是描述原子状态的好量子数.在没有外磁场时,具有相同$n$、$l$、$j$的状态是简并的.}
\textcolor{blue}{\keypoint{轨道量子数描述}{对于轨道量子数$l=0,1,2,3,\cdots$用S、P、D、F、$\cdots$来描述,并在它左上角用$2s+1$的数字来代表能级结构的多重数,右下角表示$j$,如氢原子基态为$^{2}S_{1/2}$}}
\keypoint{单电子原子的总磁矩}{考虑轨道磁矩、自旋磁矩、原子核磁矩(小,一般不考虑),$\bm{\mu}=\bm{\mu}_l+\bm{\mu}_s=-\frac{\mu_B}{\hbar}\left(g_l\bm{l}+g_s\bm{s}\right)$}
\textcolor{blue}{\keypoint{朗德$g$因子}{$g=1+\frac{\bm{j}^2-\bm{l}^2+\bm{s}^2}{2\bm{j}^2}$}}
\keypoint{用原子有效磁矩来代替总磁矩}{$\mu_j=-g\frac{\mu_B}{\hbar}\bm{j}$,其中$g=1+\frac{j(j+1)+s(s+1)-l(l+1)}{2j(j+1)}$}
\textcolor{blue}{\keypoint{有效磁矩}{
	$$\left\{
		\begin{array}{l}
		\mu_j=g\sqrt{j(j+1)}\mu_B\\
		\mu_{jz}=-gm_j\mu_B\qquad\mbox{其中,}m_j=-j,\cdots,j\\
		\end{array}
	\right. $$
}}
\subsection{单电子原子能级的精细结构}
\subsubsection{精细结构——由相对论效应引起}
\textcolor{blue}{\keypoint{能级分裂}{
	考虑相对论效应和上一节的自旋轨道耦合能而导致.因为$j$取值有两种情况,所以劈裂为两个能级.
	$$\Delta E_n=-E_n\frac{\alpha^2Z^2}{n^2}\left(\frac{3}{4}-\frac{n}{j+1/2}\right)$$
	值得注意的是,尽管两方面的修正都和$l$有关;但总的修正量只和$j,n$有关,而和$l$无关.但实际上还有下节讲的超精细结构.
}}
\keypoint{精细结构下选择定则}{
	$$
	\left\{
	\begin{array}{l}
	\Delta l=\pm1,\ \mbox{依然满足$l$的选择定则}\\
	\Delta j=0,\pm1,\ \mbox{多了$j$的选择定则}\\
	\end{array}
	\right.
	$$
}
\subsubsection{超精细结构——原子核磁矩$\bm{\mu_I}$和电子运动磁场相互作用结果}
\keypoint{原子核自旋量子数}{记为$\bm{I}$}
\keypoint{总角动量$\bm{F}$}{$\bm{F}=\bm{I}+\bm{j}$}
\keypoint{总角动量量子数$F$}{$F=(I+j),(I+j-1),\cdots,|I-j|$}

\rule{\textwidth}{0.50mm}


\section{氦原子和多电子原子}
\subsection{氦原子的能级}
\subsubsection{氦原子的光谱和能级}
\keypoint{氦原子能级特点}{
	\par \qquad1.两套能级:单态和三重态;
	\par \qquad2.基态和第一激发态之间能量差很大;
	\par \qquad3.三重态能级总低于相应单态能级;
	\par \qquad4. n=1原子不存在三重态;
	\par \qquad5.第一激发态$2^1S_0$和$2^3S_1$都是亚稳态.
}
\subsubsection{氦原子能级的简单讨论}
多电子,需要考虑电子间库伦作用,以及各种自旋轨道间相互作用.其中自旋、轨道间相互作用是磁相互作用,比静电相互作用小得多,如果只考虑主要结构(粗结构),而不涉及精细结构,则可以忽略这些较小的作用.\par
\keypoint{氦原子中位势}{$V(r_1,r_2,r_{12})=-\frac{Ze^2}{4\pi\epsilon_{0}r_2}-\frac{Ze^2}{4\pi\epsilon_{0}r_1}+\frac{e^2}{4\pi\epsilon_{0}r_{12}}$}
\keypoint{氦原子定态薛定谔方程}{$$\left[-\frac{\hbar^2}{2m_e}(\nabla_1^2+\nabla_2^2)+V(r_1,r_2,r_{12})\right]u(\bm{r_1},\bm{r_2})=Eu(\bm{r_1},\bm{r_2})$$}
\keypoint{忽略电子间库仑相互作用}{$V(r_1,r_2)=-\frac{Ze^2}{4\pi\epsilon_{0}r_2}-\frac{Ze^2}{4\pi\epsilon_{0}r_1}$}
\keypoint{每个电子相应有一组量子化能量}{$E_n=-\frac{Z^2}{n^2}\frac{e^2}{4\pi\epsilon_0\cdot2a_0}=-\frac{4}{n^2}\times13.6eV$}
\keypoint{两电子间相互作用势均值}{$\Delta E=\frac{e^2}{4\pi\epsilon_0}\left\langle \frac{1}{r_{12}}\right\rangle=\frac{5}{4}\frac{e^2}{4\pi\epsilon_0a_0}=34eV$}
\subsection{全同粒子和泡利不相容原理}
\subsubsection{全同粒子与波函数的交换对称性}
\keypoint{全同粒子}{内禀属性完全相同的粒子.内禀属性:静止质量,电荷,自旋,磁矩等.}
\keypoint{全同粒子不可分辨性}{经典物理中,可以用轨道区分它们,但量子力学中,由于测不准,轨道不具意义,因此全同粒子具有不可分辨性.}
\keypoint{全同粒子的交换对称性}{全同粒子系统中任何两个粒子的交换并不会改变这个系统的物理状态.因此有
	$$|\Psi(q_1,q_2)|^2=|\Psi(q_2,q_1)|^2
	\Rightarrow
	\Psi(q_1,q_2)=\pm\Psi(q_2,q_1)$$
}
\keypoint{全同粒子波函数必为交换对称或交换反对称}{}
\keypoint{解的线性组合构成对称/反对称解}{
	$$ \left\{
	\begin{array}{l}
	\Psi_S(q_1,q_2)=\frac{1}{\sqrt{2}}\left[\Psi_\alpha(q_1)\Psi_\beta(q_2)+\Psi_\alpha(q_2)\Psi_\beta(q_1)\right]\\
	\Psi_A(q_1,q_2)=\frac{1}{\sqrt{2}}\left[\Psi_\alpha(q_1)\Psi_\beta(q_2)-\Psi_\alpha(q_2)\Psi_\beta(q_1)\right]\\
	\end{array}
	\right.
	$$
}
\subsubsection{泡利不相容原理}
\keypoint{泡利不相容原理}{在多电子原子中,任何两个电子都不可能处在相同的量子态}
\keypoint{确定电子状态需要的量子数}{$n,l,j,m_j$或$n,l,m_l,m_s$}
\keypoint{具有\underline{交换反对称性}的全同粒子处在相同状态概率为0}{
	$$	\Psi_A(q_1,q_2)=\frac{1}{\sqrt{2}}\left[\Psi_\alpha(q_1)\Psi_\alpha(q_2)-\Psi_\alpha(q_2)\Psi_\alpha(q_1)\right]=03
	$$
}
\keypoint{不相容原理的另一表达}{多电子系统的波函数一定是反对称的}
\keypoint{反对称波函数的系统}{自旋量子数为半整数($\frac{1}{2}\hbar,\frac{3}{2}\hbar,\cdots$)的粒子组成的全同粒子系统.如电子和质子组成的系统.}
\keypoint{对称波函数的系统}{自旋量子数为半整数($0,1\hbar,2\hbar,\cdots$)的粒子组成的全同粒子系统.如光子.}
\subsubsection{交换效应}
\keypoint{交换效应}{两个电子自旋相反时,出现在空间同一位置概率很大;相同时,则概率很小.}
\subsection{多电子原子的电子组态}
\keypoint{多电子原子系统哈密顿算符}{$H=T+V=\sum\limits_{i}^{}\left(-\frac{\hbar^2}{2m_e}\nabla_i^2\right)-\sum\limits_{i}^{}\frac{Ze^2}{4\pi\epsilon_{0}r_i}+\sum\limits_{i>j}^{}\frac{e^2}{4\pi\epsilon_{0}r_{ij}}+V_{SO}+V_{SS}+V_{ll}+\cdots$,其中第一项是电子动能;第二项是电子和原子核库仑势;第三项是电子间库仑势;$v_{SO}$是电子自旋-轨道磁相互作用势;$V_{SS},\ V_{ll}$是电子间自旋-自旋相互作用势和轨道-轨道相互作用势及其他一些更弱的相互作用势.}
\keypoint{中心力场近似}{将原子中的每一个电子看成是在原子核势场及其余$(N-1)$个电子的球对称平均场$S(r_i)$中运动.对应薛定谔方程和氢原子的薛定谔方程相似.}
\keypoint{第$i$个电子的空间波函数}{可用量子数$n_i,l_i,m_{li}$描述,取值为:
	$$\left\{
		\begin{array}{l}
		n_i=1,2,3,\cdots\\
		l_i=0,1,2,\cdots,n_i-1\\
		m_{li}=-l_i,-l_i+1,\cdots,l_i-1,l_i\\
		\end{array}
	\right. $$
}
\textcolor{blue}{\keypoint{电子组态记法}{只要确定了每个电子的$(n_i,l_i)$,能量也就确定了.一般用光谱学记法:$n_il_i^N$,右上角的$N$表示电子个数.例如基态氦原子$1s^2$.}}
\subsection{原子的壳层结构和元素周期表}
\subsubsection{原子中电子的壳层结构}
\keypoint{同一壳层电子}{具有相同$n$的值}
\keypoint{支壳层}{不同的$l$量子数}
\keypoint{支壳层最大可容纳电子数}{$N_l=2(2l+1)$}
\textcolor{blue}{\keypoint{主壳层最大可容纳电子数}{$N_n=\sum\limits_{l=0}{n-1}2(2l+1)=2n^2$}}
\keypoint{能量次序}{由$E_n=-\frac{1}{2}\alpha^2m_ec^2\frac{Z^{*2}}{n^2}$知$n$越大,能量越大;而同一壳层内,$l$越大,能量越大.但会有相邻壳层的支壳层能级交错现象.}
\textcolor{blue}{\keypoint{闭合壳层的性质}{闭合壳层内磁量子数$M_L$和$M_S$均为0,因此角动量为0,不必考虑.如碱金属的角动量都由最外层$s$电子提供,基态时原子态为$^2S_{1/2}$}}
\subsection{多电子原子的原子态和能级}
讨论剩余库仑排斥作用和磁相互作用对能级和原子态的影响.最主要的是:1.电子的剩余库仑排斥作用;2.电子的自选和轨道相互作用.
\subsubsection{LS耦合——自旋较强,电子间轨道作用也较强}
\textbf{1.剩余库仑相互作用引起的能级分裂}\par
\keypoint{总轨道角动量$\bm{L}$}{总轨道角动量守恒}
\keypoint{总轨道角动量的量子数}{$L=l_1+l_2,l_1+l_2-1,\cdots,|l_1-l_2|$}
\textcolor{blue}{\keypoint{总轨道角动量大小}{$\bm{L}^2=L(L+1)\hbar^2$}}
\textcolor{blue}{\keypoint{$\bm{L}$的$z$分量}{$L_z=M_L\hbar$}}
\keypoint{$L$的相应磁量子数$M_L$}{$M_L=L,L-1,\cdots,-L$}
\keypoint{总自旋角动量$\bm{S}$}{总自旋角动量守恒}
\keypoint{总自旋角动量的量子数}{$S=s_1+s_2,s_1+s_2-1,\cdots,|s_1-s_2|$,由于$s=\pm\frac{1}{2}$,因此$S=0,1$}
\textcolor{blue}{\keypoint{总自旋角动量大小}{$\bm{S}^2=S(S+1)\hbar^2$}}
\textcolor{blue}{\keypoint{$\bm{S}$的$z$分量}{$S_z=M_S\hbar$}}
\keypoint{$S$的相应磁量子数$M_S$}{$M_S=S,S-1,\cdots,-S$}
\keypoint{$V$个价电子的原子的状态的好量子数}{$(n_1,l_1),(n_2,l_2),\cdots,(n_v,l_v),L,S,M_L,M_S$}
\keypoint{剩余库仑相互作用引起的分裂简并度}{$(2L+1)(2S+1)$}
\jumpline
\keypoint{总角动量$\bm{J}$}{$\bm{J}=\bm{L}+\bm{S}$}
\keypoint{总角动量大小}{$\bm{J}^2=J(J+1)\hbar^2$, $J=L+S,L+S-1,\cdots,|L-S|$}
\keypoint{总角动量$z$分量}{$J_z=M_J\hbar$}
\keypoint{好量子数变为}{$(n_1,l_1),(n_2,l_2),\cdots,(n_v,l_v),L,S,J,M_J$}
\textcolor{blue}{\keypoint{LS耦合后原子态}{$n^{2s+1}L_J$}}
\textcolor{blue}{\keypoint{LS耦合,自旋-轨道相互作用项}{
	\begin{align*}
	H'_2&=\zeta(L,S)\bm{L\cdot S}\\
		&=\frac{1}{2}\zeta(L,S)[J(J+1)-L(L+1)-S(S+1)]\hbar^2
	\end{align*}
}}
\textcolor{blue}{\keypoint{朗德间隔定则}{$E_{J+1}-E_J=\hbar^2\zeta(L,S)(J+1)$,同一多重态:具有相同的$L,S$状态}}
%\rule{\textwidth}{0.35mm}
\textbf{2.等效电子组成的原子态}\par
由于电子不再是非等效的,要考虑泡利不相容原理.\par
\keypoint{等效电子/同科电子}{$n$和$l$的量子数都相同的电子}
\keypoint{原子态数计算示例}{
	列表如下:\par
	\begin{table}[H]
		\footnotesize
		\begin{tabular}{|c|c|c|c|c|c|c|c|}
			\hline 
			& $m_{l_2}$ & 1 & 1 & 0 & 0 & -1 & -1 \\ 
			\hline 
			$m_{l_1}$ & \diagbox{$m_{s_1}$}{$m_{s_2}$} & $\frac{1}{2}$ & $-\frac{1}{2}$ & $\frac{1}{2}$ & $-\frac{1}{2}$ & $\frac{1}{2}$ & $-\frac{1}{2}$ \\ 
			\hline 
			1 & $\frac{1}{2}$ &  &  &  &  &  &  \\ 
			\hline 
			1 & $-\frac{1}{2}$ & $(1^-,1^+)$ &  &  &  &  &  \\ 
			\hline 
			0 & $\frac{1}{2}$ & $(0^+,1^+)$ & $(0^+,1^-)$ &  &  &  &  \\ 
			\hline 
			0 & $-\frac{1}{2}$ & $(0^-,1^+)$ & $(0^+,1^-)$ & $(0^-,0^+)$ &  &  &  \\ 
			\hline 
			-1 & $\frac{1}{2}$ & $(-1^+,1^+)$ & $(-1^+,1^-)$ & $(-1^+,0^+)$ & $(-1^+,0^-)$ &  &  \\ 
			\hline 
			-1 & $-\frac{1}{2}$ & $(-1^-,1^+)$ & $(-1^+,1^-)$ & $(-1^-,0^+)$ & $(-1^+,0^-)$ & $(-1^-,-1^+)$ &  \\ 
			\hline 
		\end{tabular} 
	\end{table}
	另一半边由于对称不可区分,去掉.对角线由于泡利不相容原理,去掉.于是剩下左下三角.\par
	共剩下15个.\par
	把表简化:\par
	\begin{table}[H]
		\footnotesize
		\begin{tabular}{|c|c|c|c|}
			\hline 
			\diagbox{$M_S$}{$M_L$} & 2 & 1 & 0 \\ 
			\hline 
			1 &  & \tabincell{c}{1:\\$(1^+,0^+)$} & \tabincell{c}{1:\\$(-1^+,1^+)$} \\ 
			\hline 
			0 & \tabincell{c}{1:\\$(1^+,1^-)$} & \tabincell{c}{2:\\ $(0^+,1^-)$\\$(0^-,1^+)$} & \tabincell{c}{3:\\$(-1^-,1^+)$\\$(0^+,0^-)$\\$(-1^+,1^-)$} \\ 
			\hline 
		\end{tabular}
	\end{table}
	此时有:L=2时,S=0;L=1时,S=1;L=0时,S=0.(永远取最大的,先$L$后$S$)\par
	即得:$^1D_2,^3P_{2,1,0},^1S_0$三项,五能级(由于$M$取值的对称性,另一半边也是得到这样的结果,不必多加考虑)
}
\textcolor{blue}{\keypoint{两个等效电子(同科电子)耦合原子态的简单规则}{可能形成的原子态一定是$L+S=$偶数的状态}}
\keypoint{解题方法}{\par
	等效电子:上述列表法\par
	多个非等效原子:两个两个耦合\par
	等效+非等效:先算等效电子的,再去耦合非等效的\par
}
\textbf{3.原子基态的量子数}\par
\textcolor{blue}{\keypoint{洪德定则}{给定电子组态能量最低的原子态的$L$和$S$这样确定:先取$S$最大值;这样的条件下再取$L$最大值;最后考虑自旋-轨道耦合,价电子数$\nu<(2l+1)$,即小于半满支壳层的电子数多重态,则$J$越小能量越低,是正常次序,若大于则$J$越大能量越低.}}
\keypoint{帕邢-巴克效应(强磁场)}{谱线分裂$\Delta E=(\Delta M_L+2\Delta M_S)\mu_BB_0$}
\keypoint{上述效应在强磁场下选择定则}{
	$$ 
	\left\{
	\begin{array}{l}
	\Delta M_S=0\\
	\Delta M_L=0,\pm1\\
	\end{array}
	\right.
	$$
}
\subsubsection{jj耦合——自旋-轨道相互作用大于剩余静电相互作用}
\keypoint{jj耦合}{
	$$
	\begin{array}{l}
	j_1=l_1+s,l_1-s=l_1+1/2,l_1-1/2\\
	j_2=l_2+s,l_2-s=l_2+1/2,l_2-1/2\\
	J=j_1+j_2,j_1+j_2-1,\cdots,|j_1-j_2|\\
	\end{array}
	.$$
}
\keypoint{jj耦合原子态表达方式}{$(j_1,j_2)_{J\mbox{所有取值}}$}
\subsection{多电子原子的光谱}
\subsubsection{选择定则}
\keypoint{选择定则}{$\Delta\left(\sum\limits_{i}l_i\right)=\pm1$}
\textcolor{blue}{\keypoint{电偶极跃迁选择定则}{
	$$
	\left\{
	\begin{array}{l}
	\mbox{LS耦合}
	\left\{
	\begin{array}{l}
	\Delta S=0\\
	\Delta L=0,\pm1\\
	\Delta J=0,\pm1(except\ J=0\rightarrow J=0)\\
	\Delta M_J=0,\pm1\\
	\end{array}
	\right.\\
	\mbox{jj耦合}
	\left\{
	\begin{array}{l}
	\Delta j=0,\pm1\\
	\Delta J=0,\pm1(except\ J=0\rightarrow J=0)\\
	\Delta M_J=0,\pm1\\
	\end{array}
	\right.
	\end{array}
	\right.
	$$
}}


\end{document}








